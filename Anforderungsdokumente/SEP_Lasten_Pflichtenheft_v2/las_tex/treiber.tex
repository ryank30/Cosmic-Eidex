\chapter{Projekttreiber}

\section{Vorbemerkung}

Das Lasten- und Pflichtenheft wird als technische Beschreibungen gesehen, entsprechend sind zu besseren Übersichtlichkeit die Vorlagen nicht gegendert. Es bleibt den einzelnen Gruppen vorbehalten, ob sie Dokumente, Dokumentation und GUI gendern oder nicht (fließt entsprechend nicht in die Bewertung mit ein).

\section{Projektziel}

Im Rahmen des Software-Entwicklungs-Projekts {\the\year} soll ein einfach zu bedienendes Client-Server-System zum Spielen von \textit{Cosmic Eidex} über ein Netzwerk implementiert werden. Die Benutzeroberfläche soll intuitiv bedienbar sein.

\section{Stakeholders}

\newcounter{sh}\setcounter{sh}{10}

\begin{description}[leftmargin=5em, style=sameline]
	
	\begin{lhp}{sh}{SH}{sh:Spieler}
		\item [Name:] Spieler
		\item [Beschreibung:] Menschliche Spieler.
		\item [Ziele/Aufgaben:] Das Spiel zu spielen.
	\end{lhp}
	
	\begin{lhp}{sh}{SH}{bsh:Spieler}
		\item [Name:] Eltern
		\item [Beschreibung:] Eltern minderjähriger Spieler.
		\item [Ziele/Aufgaben:] Um die Spieler zu kümmern, indem Eltern Spielzeit begrenzen wollen und zugriff auf sensible Inhalte begrenzen.
	\end{lhp}
	
	\begin{lhp}{sh}{SH}{bsh:gesetzgeber}
		\item [Name:] Gesetzgeber
		\item [Beschreibung:] Das Amt für Jugend und Familie.
		\item [Ziele/Aufgaben:] Die Rechte der Spieler zu schützen und zu gewähren, indem er Gesetze erstellt.
	\end{lhp}
	
	\begin{lhp}{sh}{SH}{bsh:investor}
		\item [Name:] Investoren (nur für Beispielzwecken)
		\item [Beschreibung:] Parteien, die das Finanzmittel für die Entwicklung des Systems bereitstellen.
		\item [Ziele/Aufgaben:] Gewinn zu ermitteln, indem das System an Endverbraucher verkauft wird.
	\end{lhp}
	
	\begin{lhp}{sh}{SH}{bsh:betreuer}
		\item [Name:] Betreuer
		\item [Beschreibung:] HiWis, die SEP Projektgruppen betreuen.
		\item [Ziele/Aufgaben:] Das Entwicklungsprozess zu betreuen, zu überwachen und teilweise zu steuern als auch die Arbeit der Projektgruppen abzunehmen sowie den Studenten im Prozess Hilfe zur Verfügung zu stellen. 
	\end{lhp}
	
	\begin{lhp}{sh}{SH}{bsh:prof}
		\item [Name:] apl. Prof. Dr. Achim Ebert
		\item [Beschreibung:] Der Leiter des gesamten Projekts
		\item [Ziele/Aufgaben:] Die HiWis anleiten, ihnen sagen, worauf sie bei den Projekten achten sollen und ihnen helfen, wenn sie Probleme haben.
	\end{lhp}
		
\end{description}

\section{Aktuelle Lage}

Aktuell wird das Spiel Cosmic Eidex persönlich gespielt, wo die Spieler zusammen in einem Raum sein müssen und die Karten wie bei jedem anderen Kartenspiel ausgeteilt werden und jeder Spieler die Spielregeln kennen sollte. Das Problem ist, dass die Spieler alle im selben Raum sein müssen, um das Spiel zu spielen und wenn das Regelbuch nicht direkt zugänglich ist, muss man darauf vertrauen, dass die anderen Spieler die Regeln auswendig kennen. Das Ziel dieses Projekts ist es, eine digitale Version von Cosmic Eidex zu erstellen, die es den Benutzern ermöglicht, gemeinsam online zu spielen und mit Leuten aus verschiedenen Orten zu interagieren. Der Benutzer kann sich in das Spiel einloggen und Spielräumen beitreten, die in der Lobby sichtbar sind. In diesen Spielräumen können die Spieler dann ein Spiel beginnen, wenn die erforderliche Anzahl von Spielern erreicht ist, um das Spiel zu starten. Sowohl in der Lobby als auch in den Spielräumen können die Spieler über die Chat-Funktion miteinander kommunizieren. Darüber hinaus wird ein Bot-Spieler hinzugefügt, falls der Spieler selbst üben und seine Kenntnisse des Spiels verfeinern möchte. Und die Eltern profitieren davon, dass sie sehen können, wie lange die Kinder spielen und sicher sind, dass sie nicht zu lange spielen. 