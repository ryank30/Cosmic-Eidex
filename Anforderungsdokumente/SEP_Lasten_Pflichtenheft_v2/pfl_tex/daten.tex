\subsection{Data-Dictionary}

\paragraph{Fachlich-Bedingte Daten}

\begin{itemize}
\item[\ref{daten:benutzername}] Benutzername* = string
\item[\ref{daten:passwort}] Passwort* = string
\item[\ref{daten:gw_punkte}] Gewinnpunkte im Spiel = integer
\item[\ref{daten:siege}] Siege* = integer
\item[\ref{daten:gamestate}] Spielzustand = object
\item[\ref{daten:msg}] Chatnachricht = string
\item[\ref{daten:gameroom_name}] Spielraumname* = string
\item[\ref{daten:gameroom_password}] Spielraumpasswort* = string
\item[\ref{daten:gameroom_pop}] Spielraumfüllung* = integer
\end{itemize}

\paragraph{Technisch-Bedingte Daten}

\newcounter{pd}\setcounter{pd}{10}

\begin{description}[leftmargin=5em, style=sameline]
\begin{lhp}{pd}{PD}{datadict:benutzername}
\item [Name:] Zugriffsdaten*
\item [Motivation:] Vereinfachung von Classendiagramm (wirklich so?) sowie Vereinfachung von Anforderungen und Dokumentation.
\item [Data-Dictionary Ausdruck]: Zugriffsdaten = Benutzername + Passwort
\item [Relevante Systemfunktionen:]  \ref{funk:spielverw}, \ref{funk:zugriff}
\end{lhp}

\begin{lhp}{pd}{PD}{datadict:gameinfo}
    \item [Name:] Spielinformation
    \item [Motivation:] Zusammenfassung von Spielraumname, Spielraumpasswort und Spielraumfüllung zur besseren Übersicht
    \item [Data-Dictionary Ausdruck]: Spielinformation = Spielraumname + Spielraumpasswort + Spielraumfüllung
    \item [Relevante Systemfunktionen:]  \ref{funk:spielraum}
\end{lhp}

\begin{lhp}{pd}{PD}{datadict:spielstatus}
    \item [Name:] Spielstatus
    \item [Motivation:] Vereinheitlichung von spielbezogenen Daten
    \item [Data-Dictionary Ausdruck]: Spielstatus = Spielzustand + Gewinnpunkte
    \item [Relevante Systemfunktionen:]  \ref{funk:spielverw}
\end{lhp}

\begin{lhp}{pd}{PD}{datadict:chatinfo}
    \item [Name:] Chatinformationen
    \item [Motivation:] Zusammenfassung von Nachrichten für Chatverwaltung
    \item [Data-Dictionary Ausdruck]: Chatinformationen = Chatnachricht
    \item [Relevante Systemfunktionen:]  \ref{funk:chat}
\end{lhp}

\begin{lhp}{pd}{PD}{datadict:handkarten}
    \item [Name:] HandkartenInformation
    \item [Motivation:] Die Anzahl der Handkarten eines Spielers soll separat erfasst werden, um Spielverlauf korrekt zu steuern.
    \item [Data-Dictionary Ausdruck:] HandkartenInformation = handkartenAnzahl
    \item [Relevante Systemfunktionen:] \ref{funk:spielverw}, \ref{funk:bots}
\end{lhp}

\begin{lhp}{pd}{PD}{datadict:trumpf}
    \item [Name:] TrumpfInformation
    \item [Motivation:] Die momentan gültige Trumpffarbe soll zentral gespeichert werden, um Regeln korrekt anzuwenden.
    \item [Data-Dictionary Ausdruck:] TrumpfInformation = trumpfFarbe
    \item [Relevante Systemfunktionen:] \ref{funk:spielverw}, \ref{funk:spielraum}, \ref{funk:bots}
\end{lhp}

\begin{lhp}{pd}{PD}{datadict:stich}
    \item [Name:] StichInformation
    \item [Motivation:] Die aktuell gespielten Karten im Stich sollen einheitlich verwaltet werden, um den Gewinner zu ermitteln.
    \item [Data-Dictionary Ausdruck:] StichInformation = aktuellerStich
    \item [Relevante Systemfunktionen:] \ref{funk:spielverw}, \ref{funk:bots}
\end{lhp}

\begin{lhp}{pd}{PD}{datadict:druecker}
    \item [Name:] DrueckerPunkteInformation
    \item [Motivation:] Die verdeckten Karten und ihre Punkte sollen separat gespeichert werden, um die Endwertung zu ermöglichen.
    \item [Data-Dictionary Ausdruck:] DrueckerPunkteInformation = drueckerPunkte
    \item [Relevante Systemfunktionen:] \ref{funk:spielverw}, \ref{funk:bestenliste}, \ref{funk:bots}
\end{lhp}

\begin{lhp}{pd}{PD}{datadict:spezialfaehigkeit}
    \item [Name:] SpezialfaehigkeitInformation
    \item [Motivation:] Die Spezialfähigkeit des Spielers soll eindeutig zugewiesen und bei Spielentscheidungen berücksichtigt werden.
    \item [Data-Dictionary Ausdruck:] SpezialfaehigkeitInformation = cosmicCharacter
    \item [Relevante Systemfunktionen:] \ref{funk:spielverw}, \ref{funk:bots}, \ref{funk:chat}
\end{lhp}

\end{description}

